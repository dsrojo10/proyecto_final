\documentclass{article}
\usepackage[utf8]{inputenc}
\usepackage[spanish]{babel}
\usepackage{listings}
\usepackage{graphicx}
\graphicspath{ {images/} }
\usepackage{cite}

\begin{document}

\begin{titlepage}
    \begin{center}
        \vspace*{1cm}
            
        \Huge
        \textbf{Proyecto Final}
            
        \vspace{0.5cm}
        \LARGE
        Los primeros pasos
            
        \vspace{1.5cm}
            
        \textbf{David Santiago Rojo C.\\
        Juan Diego Sanchez.}
            
        \vfill
            
        \vspace{0.8cm}
            
        \Large
        Departamento de Ingeniería Electrónica y Telecomunicaciones\\
        Universidad de Antioquia\\
        Medellín\\
        Marzo de 2021
            
    \end{center}
\end{titlepage}

\tableofcontents
\newpage


\section{Sección introductoria}\label{intro}
El proyecto final es una parte fundamental de este curso, ya que en el se presenta la posibilidad de demostrar y aplicar todo lo estudiado en este y en cursos previos de lógica y programación. Este primer acercamiento trata de las ideas iniciales para abarcar el proyecto, las cuales irán siendo desarrolladas con el pasar del curso y a las cuales se irán agregando y/o modificando ideas.

\section{Idea: BATTLE-AIRCRAFT} \label{contenido}
•	Juego tipo “Arcade”.\cite{euronics} \\

•	Avión de combate que va avanzando y debe ir superando los diferentes obstáculos que se atraviesan para seguir con vida y llegar a la meta final del nivel.\\

•	Niveles cortos, pero con obstáculos variados.\\

•	Aparición de estrellas o diferentes figuras que aumentan o disminuyen la velocidad con la que el juego avanza, cambian el color del fondo o incluso la forma de los obstáculos.\\

•	Opción multijugador (por turnos) al terminar el nivel o al morir seguirá el turno del siguiente jugador, al finalizar los mundos al perder todas las vidas ambos jugadores se mostrarán las estadísticas, el ganador será el que haya recorrido todos los niveles en el menor tiempo posible, el cual será contabilizado desde que inicia el nivel hasta atravesar la meta.\\

•	En determinados momentos será posible disparar para destruir los obstaculos\\

•	Los controles del juego serán a través del mouse.\\ 



\bibliographystyle{IEEEtran}
\bibliography{references}

\end{document}
